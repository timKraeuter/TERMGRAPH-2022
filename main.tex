\documentclass[adraft, copyright, creativecommons]{eptcs} % TODO: Change to submission at the end
\providecommand{\event}{TERMGRAPH 2022} % Name of the event you are submitting to
\usepackage{breakurl}             % Not needed if you use pdflatex only.
\usepackage{underscore}           % Only needed if you use pdflatex.

\title{BPMN semantics formalization and \\ model checking using graph grammars}
\author{Tim Kräuter
\institute{NICTA\\ Sydney, Australia}
\institute{School of Computer Science and Engineering\\
University of New South Wales\thanks{A fine university.}\\
Sydney, Australia}
\email{tkra@hvl.no}
\and
Co Author \qquad\qquad Yet S. Else
\institute{Stanford Univeristy\\
California, USA}
\email{\quad is@gmail.com \quad\qquad somebody@else.org}
}
\def\titlerunning{BPMN semantics formalization and model checking using graph grammars}
\def\authorrunning{Kräuter \textit{et al.}}
\begin{document}
\maketitle

% Maximum 8 pages for the first extended abstract!

\begin{abstract}
TBD abstract
\end{abstract}

\section{Introduction}
\begin{itemize}
    \item State goals: Semantics formalization/reference implementation and model checking.
\end{itemize}

\section{BPMN semantics formalization}
\begin{itemize}
    \item General approach: Model transformation from any bpmn file to a graph grammar. Rules are generated specifically for each file.
    \item Go through the BPMN spec and explain the formalization similar to the structure in \cite{vangorpVisualTokenbasedFormalization2013}.
    \item Talk about best practices which help explain why some elements have not been formalized yet (Problems with the inclusive gateway).
\end{itemize}

\section{Related work}
\begin{itemize}
    \item Compare extend of the formalization in this paper with other papers formalizing BPMN semantics (Make a big table with the futures and different approaches).
    \item Other formalizations are given in: \cite{vangorpVisualTokenbasedFormalization2013} graph transformations, \cite{wongProcessSemanticsBPMN2008} CSP, \cite{dijkmanSemanticsAnalysisBusiness2008} Petri Nets and more to find ...
    \item Especially talk about \cite{vangorpVisualTokenbasedFormalization2013} since it also used graph rewriting but differently: fixed set of rules vs. generated set of rules by a model transformation from bpmn + differences in the resulting tools (we read bpmn files directly).
    \item At the end we should cover more of the semantics than any currently given approach.
    \item Might also be interesting to compare use to Camunda or the bpmn-js token simulator. However these projects have different goals.
\end{itemize}

\section{Model checking BPMN}
\begin{itemize}
    \item Graph grammar generates state space.
    \item Graph conditions can be used to add atomic propositions to the state space.
    \item Temporal Logic can be used for model checking with the defined atomic propositions.
    \item LTL properties can for example be checked in Groove.
\end{itemize}
\section{Related work}
\begin{itemize}
    \item Talk about the tool prototype.
    \item No installation should be needed.
\end{itemize}

\section{Conclusion}

\bibliographystyle{eptcs}
\bibliography{bib} % TODO: Should be bib.
\end{document}
